\section{Introduction}
The advent of Large Language Models (LLMs) has triggered a paradigm shift in how individuals seek and obtain technical information. Stack Overflow, as the premier programming question-and-answer platform, has long been the go-to resource for developers facing coding challenges. However, with the public release of ChatGPT in November 2022, developers gained access to an AI assistant capable of providing immediate, contextual programming guidance—potentially disrupting established knowledge-seeking patterns on specialized forums.\\

This paper investigates how ChatGPT's introduction has altered the landscape of programming questions on Stack Overflow, employing natural language processing (NLP) techniques to identify and quantify changes in question content, complexity, and topical focus. We first establish the causal impact of ChatGPT on Stack Overflow question volumes using difference-in-differences methods, then leverage this foundation to conduct a comprehensive NLP analysis of the changing patterns in developer queries.\\

Our dataset spans January 2021 to March 2024, encompassing over 1.3 million questions from Stack Overflow. The initial dataset size was around 100 GB, and still 2.5 GB of data, or roughly 4 million rows, for our selected time frame, including all questions. Thus, we focused on scripting languages (i.e., JavaScript, Python, R, and PHP) \parencite{stack_overflow_tags_nodate}.\\

The focus on scripting languages has also several other advantages, i.e.: (1) We effectively reduce the dataset size and thus also pre-processing times significantly. (2) The selected languages are also the top-contributing programming languages on the platform itself and thus representative of the overall platform trends. (3) The chosen languages are those which, assumedly, saw the largest impact as they are high-level languages that earlier ChatGPT versions were especially trained on and sound in. For the causal analysis component, we incorporate data from four non-programming Stack Exchange forums (Mathematics, Physics, Superuser, and AskUbuntu) as control units (approx. 0.5 million questions) without any exclusions \parencite{internet_archive_stackexchange_2024}.

%%%%%%%%%%%%%%%%%%%%%%%%%%%%%%%%%%%%%%%%%%%%%%%%%%%%%%%%%%%%%%%%%%%%%%%%%%%%%%%%%%%%%%%%%%%%%%%%

\subsection{Research Questions and Approach}
Thus, our primary research questions ask: 
\begin{enumerate}
    \item To what extent has ChatGPT's introduction causally affected question volume on Stack Overflow?
    \item How has the' nature, complexity, and topical distribution of questions changed post-ChatGPT?
    \item Can NLP techniques effectively detect and characterize these shifts in question patterns?
\end{enumerate}

We approach these questions through a two-stage methodology. First, we establish causality through a Synthetic Difference-in-Differences (SDID) framework, quantifying the volumetric impact while controlling for temporal trends following \textcite{arkhangelsky_synthetic_2021} and \textcite{clarke_synthetic_2023}. This causal foundation motivates our core NLP analysis, employing \textit{[NLP METHOD PLACEHOLDER]} to analyze the content of Stack Overflow questions before and after ChatGPT's release.