\section{Introduction}
The advent of Large Language Models (LLMs) has triggered a paradigm shift in how individuals seek and obtain technical information. Stack Overflow, as the premier programming question-and-answer platform, has long been the go-to resource for developers facing coding challenges. However, with the public release of ChatGPT in November 2022, developers gained access to an AI assistant capable of providing immediate, contextual programming guidance—potentially disrupting established knowledge-seeking patterns on specialized forums.\\

This paper investigates how ChatGPT's introduction has altered the landscape of programming questions on Stack Overflow, employing natural language processing (NLP) techniques to identify and quantify changes in question content, complexity, and topical focus. We first establish the causal impact of ChatGPT on Stack Overflow question volumes using difference-in-differences methods, then leverage this foundation to conduct a comprehensive NLP analysis of the changing patterns in developer queries.

\subsection{Research Questions and Approach}
Our primary research questions ask: 
\begin{enumerate}
    \item To what extent has ChatGPT's introduction causally affected question volume on Stack Overflow?
    \item How has the' nature, complexity, and topical distribution of questions changed post-ChatGPT?
    \item Can NLP techniques effectively detect and characterize these shifts in question patterns?
\end{enumerate}

We approach these questions through a two-stage methodology. First, we establish causality through a Synthetic Difference-in-Differences (SDID) framework, quantifying the volumetric impact while controlling for temporal trends. This causal foundation motivates our core NLP analysis, employing \textit{[NLP METHOD PLACEHOLDER]} to analyze the content of millions of Stack Overflow questions before and after ChatGPT's release.\\

Our dataset spans January 2021 to March 2024, encompassing over \textit{[NUMBER]} questions from Stack Overflow. For the causal analysis component, we incorporate data from four non-programming Stack Exchange forums (Mathematics, Physics, Superuser, and AskUbuntu) as control units. This comprehensive dataset allows us to both establish causality and conduct in-depth textual analysis to characterize the changing nature of developer queries in the age of AI assistants.

\section{Causal Impact Analysis}
Before diving into our primary NLP analysis, we first establish the causal impact of ChatGPT on Stack Overflow question volumes. This section briefly outlines our causal inference methodology and findings, which provide critical context for interpreting the subsequent NLP results.

\subsection{Synthetic Difference-in-Differences Methodology}
To identify the causal impact of ChatGPT on Stack Overflow question volumes, we employ a Synthetic Difference-in-Differences (SDID) approach. This methodology combines the strengths of traditional difference-in-differences and synthetic control methods, allowing us to construct a credible counterfactual for Stack Overflow in the absence of ChatGPT.\\

The SDID estimator constructs a synthetic version of Stack Overflow by optimally weighting other Stack Exchange forums (Mathematics, Physics, Superuser, and Ask Ubuntu) and pre-treatment periods. The selection of forums as control units was strategically motivated by several considerations:
\begin{enumerate}
    \item These Stack Exchange forums represent technical knowledge domains with structured question patterns similar to Stack Overflow, yet they address distinct subject matters that were less effectively handled by early ChatGPT versions (3.5). While ChatGPT demonstrated strong capabilities in programming tasks from its initial release, it exhibited notable limitations in advanced mathematics, physics reasoning, and system-specific troubleshooting—areas central to our control forums. 
    \item These forums maintain sufficient question volumes to provide statistical power while exhibiting pre-treatment correlation with Stack Overflow question patterns (correlation coefficients between 0.76-0.87), suggesting similar responsiveness to seasonal trends and external factors affecting forum usage. 
    \item These forums occupy adjacent positions in the broader STEM knowledge ecosystem, making them vulnerable to similar secular trends in online knowledge-seeking behavior while remaining differentially exposed to the treatment effect of interest, thus satisfying the key assumptions required for synthetic control methods.
\end{enumerate}
Formally, the SDID estimator can be expressed as:

\begin{equation}
\hat{\tau}_{SDID} = \sum_{t=T_0+1}^T \lambda_t \left( Y_{1t} - \sum_{j=2}^J \omega_j Y_{jt} \right) - \sum_{t=1}^{T_0} \lambda_t \left( Y_{1t} - \sum_{j=2}^J \omega_j Y_{jt} \right)
\end{equation}

where $Y_{jt}$ represents the log-transformed question count for forum $j$ at time $t$, $\omega_j$ are unit weights, $\lambda_t$ are time weights, $T_0$ is the last pre-treatment period, and unit $j=1$ represents Stack Overflow. We implemented this methodology using both a custom implementation and Microsoft's \href{https://microsoft.github.io/SynapseML/docs/Overview/}{SynapseML} library to ensure robustness of our findings.

\subsection{Causal Impact Results}

Figure \ref{fig:synthetic_control} visualizes the divergence between Stack Overflow's actual question volume and its synthetic counterfactual after ChatGPT's release. The figure shows that while the synthetic control closely tracks Stack Overflow's question volume before ChatGPT's release, a gap emerges immediately after treatment.

\begin{figure}[htpb!]
    \centering
    \includesvg[width=0.75\textwidth]{imgs/synthetic_control_log.svg}
    \caption{Synthetic Control Analysis of ChatGPT's Impact on Stack Overflow}
    \label{fig:synthetic_control}
\end{figure}

Table \ref{tab:results} presents the formal estimates of ChatGPT's impact on Stack Overflow question volume using both our implementations.

\begin{table}[htpb!]
\centering
    \begin{tabular}{lrrrr}
        \toprule
        \textbf{Method} & \textbf{Treatment Effect} & \textbf{Standard Error} & \textbf{t-statistic} & \textbf{p-value} \\
        \midrule
        Custom SDID Implementation & -0.34 & 0.06 & -5.73 & $<0.0001$ \\
        SynapseML SDID & -0.30 & 0.07 & -4.50 & $<0.0001$ \\
        \bottomrule
    \end{tabular}
    \caption{Estimated Impact of ChatGPT on Stack Overflow Question Volume}
    \label{tab:results}
\end{table}

Both implementations yield significant adverse effects, with treatment effect estimates ranging from -0.30 to -0.34 in log units. This translates to approximately a 26-29\% reduction in Stack Overflow question volume attributable to ChatGPT's introduction. This substantial decline motivates our core research question: how has the nature of the remaining questions changed?

\section{Natural Language Processing Methodology}
\label{sec:nlp_methodology}

Building on the established causal impact, we now turn to our primary contribution: a comprehensive NLP analysis of how question content, complexity, and focus have evolved in response to ChatGPT's introduction. This section outlines our NLP methodology for detecting and characterizing these changes.

\subsection{Data Pre-processing}
\textit{[PREPROCESSING PLACEHOLDER - Will detail specific preprocessing steps including tokenization, stopword removal, lemmatization, etc.]}

\subsection{Topic Modeling}
\textit{[TOPIC MODELING PLACEHOLDER - Will describe topic modeling approach (e.g., LDA), parameter selection, and validation strategies]}

\subsection{Complexity Analysis}
\textit{[COMPLEXITY ANALYSIS PLACEHOLDER - Will outline methods for measuring question complexity, including readability metrics, technical vocabulary density, and code snippet analysis]}

\subsection{Classification Framework}
\textit{[CLASSIFICATION FRAMEWORK PLACEHOLDER - Will detail supervised classification approaches for identifying pre/post-ChatGPT questions and their characteristics]}

\section{Preliminary Results}
While our NLP analysis remains in progress, our causal findings suggest substantial changes in Stack Overflow usage patterns. Beyond the 26-29\% volume reduction, we observed a fundamental shift in the correlation structure between Stack Exchange forums after ChatGPT's release, as visualized in Figure \ref{fig:correlation_matrix}.

\begin{figure}[htpb!]
    \centering
    \includegraphics[width=1\textwidth]{}
    \caption{Correlation Matrices Before and After ChatGPT Release}
    \label{fig:correlation_matrix}
    % Figure 3 reference
\end{figure}

This dramatic weakening of correlations (from 0.76-0.87 to 0.18-0.65) suggests that ChatGPT has reduced question volume but potentially altered the relationship between programming questions and those in other knowledge domains. This finding motivates our hypothesis that the content and nature of Stack Overflow questions have fundamentally changed in the post-ChatGPT era -— a hypothesis we explore through our NLP analysis in the following sections.

\textit{[PRELIMINARY NLP FINDINGS PLACEHOLDER - Will include initial topic distribution shifts, complexity changes, and other preliminary textual analysis results]}