\subsection{Literature Review}
\subsubsection{General Technology Adaptation Literature}
This question can be approached through the Technology Adoption field which is the conceptual area concerned with user acceptance processes of new technologies, especially those related to Management Information Systems.

The current longest-standing and most hegemonic framework is Roger Davis' Technology Acceptance Model (TAM) which defined two main predictors of end-user attitudes towards a new technology: Perceived Usefulness (PU) and Perceived Ease of Use (PEOU). Through the later empirical literature, PU has been found to have the strongest relationship with positive attitudes towards technology. PEOU's predictive ability, on the other hand, has always been found to be weaker, and also, been found to be smaller and smaller as new studies have been made. 

Through the late 80s and 90s, many authors identified and suggested factors to extend Davi's model. Through an extensive revision of the literature, Kelly et al. find "trust" and "attitudes" to be the most widely applied in empirical studies. "Trust" refers to both trust in the technology itself and trust in the provider and can be both a predictor of PU or a direct predictor of acceptance. "Attitudes" refers to the subjective sentiment towards the technology and can both influence and be influenced by PU and PEOU and, therefore, act as a direct predictor of acceptance in case of the latter. 

In 2003, a formal expansion of the TAM, the Unified Theory of Acceptance and Use of Technology (UTAUT), was suggested by ----. It suggested four predictors instead of two: performance expectancy, social influence, effort expectancy, and facilitating conditions. 

\subsubsection{Empirical Literature on AI Coding Assistants}