\section{Literature Review}
This question can be approached through the Technology Adoption field, which is the conceptual area concerned with user acceptance processes of new technologies, especially those related to Management Information Systems.

\subsection{General Technology Adaptation Literature}
The current longest-standing and most hegemonic framework is Roger Davis' Technology Acceptance Model (TAM) which defined two main predictors of end-user attitudes towards a new technology: Perceived Usefulness (PU) and Perceived Ease of Use (PEOU). Through the later empirical literature, PU has been found to have the strongest relationship with positive attitudes towards technology. PEOU's predictive ability, on the other hand, has always been found to be weaker, and also, been found to be smaller and smaller as new studies have been made \parencite{davis_technology_1985, davis_perceived_1989, kelly_what_2023}.  \\

In 2003, a formal expansion of the TAM, the Unified Theory of Acceptance and Use of Technology (UTAUT), was suggested by \textcite{venkatesh_user_2003}. It suggested four predictors instead of two: performance expectancy, social influence, effort expectancy, and facilitating conditions. It aimed to address the limitations of TAM by incorporating external influences and structural factors that affect technology adoption. It has since been widely applied, with researchers often modifying or extending it to better fit specific technological contexts \parencite{venkatesh_user_2003, hasan_emon_insights_2023}.\\

Throughout the literature, many authors have applied extended versions (including additional traits) of either the TAM or the UTAUT. In an extensive revision of the literature, \textcite{kelly_what_2023} find "trust" and "attitudes" to be the most widely applied in empirical studies. "Trust" refers to both trust in the technology itself and trust in the provider and can be both a predictor of PU or a direct predictor of acceptance. "Attitudes" refers to the subjective sentiment towards the technology and can both influence and be influenced by PU and PEOU and, therefore, act as a direct predictor of acceptance in case of the latter. \\

Many authors have also included personal and cultural characteristics as predictors of attitude, including age, religion, and social context, and, in relation to this, the cognitive process has emerged as a widely considered factor. The element of cognitive process includes a combination of social characteristics and past experiences with new or similar technologies that contribute to the individual's perception of themselves and of the adoption process, thus influencing intention of use. 

\subsection{Empirical Literature on AI Coding Assistants}

Across multiple empirical studies, developers report that AI programming assistants offer significant utility in their workflow, indicating high perceived usefulness. \textcite{bird_taking_2023} observed that early Copilot users found the tool helpful for a range of coding tasks beyond simple autocompletion (ibid.). Use cases observed include delegating tedious tasks such as generating unit tests, boilerplate code, and code comments as well as providing essential assistance in the use of languages or tasks which the developer is unfamiliar with \parencite{bird_taking_2023, sergeyuk_using_2025}. This had key implications on respondents' perceived productivity. \\

Moreover, their support is multipurpose and versatile, offering assistance in practically the whole spectrum of potential issues within the same conversation. Not only does it generate code but it also engages in dialogue to explain concepts or debug errors. Its conversational ability allows developers to obtain detailed explanations and alternative solutions, thereby enhancing its role as an effective answering agent (Kabir). The tool’s extensive knowledge base and real-time responsiveness enable users to address both simple and complex queries without delay  that can be immediately followed up with clarifications or further questions in a single session, therefore maximizing the possibility of obtaining an effective answer within a short period of time. \\

AI coding assistants also significantly reduce the effort required to obtain coding assistance, giving them a very high Perceived Ease of Use. Unlike traditional Q\&A forums, which often demand well-structured inquiries and involve waiting periods for responses, ChatGPT provides immediate, natural-language answers, thereby minimizing the friction of help-seeking. The conversational format not only streamlines the process but also eliminates the social risks associated with public forums, as users can ask questions privately without fear of judgment or negative feedback (ibid). Furthermore, the low learning curve associated with these tools enables even novice developers to begin using them effectively right away (Bird et al.). Overall, the high perceived ease of use, combined with the efficiency of obtaining timely answers, reinforces the adoption of AI assistants as integral components of modern development workflows. \\

Despite these strengths, limitations remain that temper the usefulness of AI assistants. Both Copilot and ChatGPT are known to produce outputs that may be incorrect or misleading. \textcite{kabir_is_2023} report that a 52\% of answers contained incorrect information  while 78\% were inconsistent with human-generated answers. Compatibility problems, internal errors, and context misunderstandings were the most reported problems \parencite{zhou_exploring_2025}. Nonetheless, ease of use and answer presentation caused developers to still use AI in most cases. \\

This caveat leads us to two different observations. Firstly, it shows how perceived usefulness is a far more important indicator of use than actual usefulness. This idea can be reinforced by the fact that initial acceptance rate of suggestions, regardless of whether such suggestions make it to the final version of the code, is the most significant predictor of self-reported productivity by coders. Secondly, this might be the main mechanism through which change of use of Stack Overflow is affected, with developers now assigning each tool a specific type of question, or changing the presentation of their questions. \\

Our thesis is that, through our language analysis on Stack Overflow questions, we will be able to see an increase in the complexity of questions asked as well as a shift in the topics, with a significant reduction in more basic and generalized tasks and an increase in more specific questions and in questions that solve the most prominent issues arising from ChatGPT answers (compatibility and context-related problems). This aligns with predictions made by some authors as well as with early observations \parencite{kabir_is_2023, sergeyuk_using_2025, zhou_exploring_2025} but these are based on qualitative analysis mainly based on personal interviews. Our work expects to provide an answer to this question that is based on quantitative analysis and large-scale direct analysis of Stack Overflow questions.