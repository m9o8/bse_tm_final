\section{Natural Language Processing Methodology}
\label{sec:nlp_methodology}

Building on the established causal impact, we now turn to our primary contribution: a comprehensive NLP analysis of how question content, complexity, and focus have evolved in response to ChatGPT's introduction. This section outlines our NLP methodology for detecting and characterizing these changes. \figureref{fig:app-process_flow} in \appref{app:procpipe} provides a sketched overview of our entire processing pipeline including file names in a flowchart.

%%%%%%%%%%%%%%%%%%%%%%%%%%%%%%%%%%%%%%%%%%%%%%%%%%%%%%%%%%%%%%%%%%%%%%%%%%%%%%%%%%%%%%%%%%%%%%%%

\subsection{Complexity Analysis}\label{subsec:comp}

Based on the pre-processed data, we construct a parsimonious complexity score for forum posts which is composed of 4 key elements: (1) title length, (2) body length, (3) number of tags and (4) length of technical expressions (i.e. code, or equation length for the respective forums)\footnote{We used different metrics appropriate to each forum's content: code length was measured for Stack Overflow, Superuser, and Askubuntu, while equation length was used for Mathematics and Physics. This distinction reflects the content patterns in our data, as Physics and Mathematics questions contained code in fewer than 5\% of posts, while the technology-focused forums rarely contained equations.}:
\begin{multline}\label{eq:cscore}
    \text{Complexity Score}_{i,t} = \\ 
    \frac{1}{4} \left( \frac{\text{TagCount}_{i,t} - \mu_{\text{TagCount}}}{\sigma_{\text{TagCount}}} + \frac{\text{TechExprLength}_{i,t} - \mu_{\text{TechExprLength}}}{\sigma_{\text{TechExprLength}}} \right. \\
    \left. + \frac{\text{BodyLength}_{i,t} - \mu_{\text{BodyLength}}}{\sigma_{\text{BodyLength}}} + \frac{\text{TitleLength}_{i,t} - \mu_{\text{TitleLength}}}{\sigma_{\text{TitleLength}}} \right)
\end{multline}
\equationref{eq:cscore} thus shows the complexity score of forum $i$ in time (week) $t$. The reason for choosing this relatively simple score is that questions often use snippets of code or equations. Thus, more sophisticated and established equation or code complexity algorithms become unusable. \tableref{fig:complex} presents the results over time:

\begin{figure}[htpb!]
    \centering
    \includesvg[width=1\linewidth]{imgs/stackoverflow_vs_rest.svg}
    \caption{Treatment vs. Control: Question complexity over time}
    \label{fig:complex}
\end{figure}

%%%%%%%%%%%%%%%%%%%%%%%%%%%%%%%%%%%%%%%%%%%%%%%%%%%%%%%%%%%%%%%%%%%%%%%%%%%%%%%%%%%%%%%%%%%%%%%%

\subsection{Term Importance Analysis}
To understand the nature of the complexity changes identified in our Synthetic DiD analysis, we examine shifts in term importance using TF-IDF (Term Frequency-Inverse Document Frequency) analysis. Unlike raw term frequency, TF-IDF weights terms based on their relative importance within documents and across the corpus, highlighting discriminative terms while down-weighting common ones. Figure \ref{fig:tfidf} shows the significant changes in term importance before and after ChatGPT's introduction. To ensure observed term changes represent genuine shifts rather than random variation, we applied statistical testing using bootstrapped confidence intervals (100 replications) and permutation tests ($\alpha = 0.05$). This approach allowed us to identify statistically significant changes in term importance that correlate with the introduction of ChatGPT.
