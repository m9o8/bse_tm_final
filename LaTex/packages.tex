%%%%%%%%%%%%%%%%%%%%%%%%%%%%%% Doc set %%%%%%%%%%%%%%%%%%%%%%%%%%%%%%%%%%%%
\def\gr{}      % Set the group number cons. across doc
\def\nass{}    % Set the assignment number cons. across doc
\def\cl{Intro to Text Mining and NLP }   % Define the class
%%%%%%%%%%%%%%%%%%%%%%%%%%%%%% Packages %%%%%%%%%%%%%%%%%%%%%%%%%%%%%%%%%%%
\usepackage[utf8]{inputenc} % Unicode characters
% \usepackage[ngerman]{babel} % New German language corrections
\usepackage[T1]{fontenc} % Correct Umlaut appearance
\usepackage{geometry}
 \geometry{
 a4paper,
 total={170mm,257mm},
 left=30mm,
 right=25mm,
 top=30mm,
 bottom=25mm
 }
\usepackage{amsmath}
\usepackage{amssymb}
\usepackage{mathtools}
\usepackage{graphicx}
\graphicspath{{imgs/}}
\usepackage{pgfplots} % Plots
\pgfplotsset{width=12cm,compat=newest}
\usepackage{xcolor}
\usepackage{multicol}
\usepackage{threeparttable} % For tablenotes environment
\usepackage{transparent}
\usepackage{svg}
\usepackage{url}
\usepackage{booktabs}
\usepackage{tabularx}
\usepackage{subcaption}
\usepackage{float}
%\usepackage{subfig}     % Easy graphics next to each other
\usepackage{enumitem}   % Allows to change enumerate labels
\usepackage{xspace}
\usepackage{fancyhdr}
\pagestyle{fancy}                    % Eigener Seitenstil
\fancyhf{}                           % Alle Kopf- und Fußzeilenfelder bereinigen
\fancyhead[L]{DSDM - BSE}            % Kopfzeile links
\fancyhead[C]{\cl}     % Zentrierte Kopfzeile
\fancyhead[R]{Final Project \nass}          % Kopfzeile rechts
\renewcommand{\headrulewidth}{0.4pt} % Obere Trennlinie
\fancyfoot[C]{\thepage}              % Seitennummer
\usepackage{pdfpages}
\usepackage{newclude}
\usepackage{hyperref} %Links, z.B. direkt zum gewünschten Kapitel springen
\hypersetup{
    pdftitle    = {\cl - Assignment \nass},
    pdfsubject  = {This is a submission in the DSDM Masters at BSE.},
    pdfauthor   = {Group\gr},
    % pdfkeywords = {Ha11, Bachelor thesis},
    pdfcreator  = {Overleaf},
    pdfstartview= FitH,        % PDF Fenster ausgefüllt
}
%%%%%%%%%%%%%%%%%%%%%%%%%%%%%%%%% Code highlighting %%%%%%%%%%%%%%%%%%%%%%%%%%%%%%%%%
\usepackage[newfloat]{minted}     % Code highlighting
\newenvironment{code}{\captionsetup{type=listing}}{}
\SetupFloatingEnvironment{listing}{name=Code}

%   Code list in content index
\renewcommand{\listoflistings}{
  \cleardoublepage
  \addcontentsline{toc}{chapter}{List of Code}
  \listof{listing}{List of Code}
}

\setminted{
    %style=monokai,           % Color scheme
    fontsize=\small,         % Text size
    %frame=lines,             % Frame style (none, lines, single)
    %framesep=2mm,           % Frame separation
    %baselinestretch=1.2,    % Line spacing
    breaklines=true,        % Enable line breaking
    linenos=true,           % Show line numbers
    %numbersep=5pt,          % Space between numbers and code
    %tabsize=4,              % Tab size
    autogobble=true,        % Remove common leading whitespace
    %bgcolor=lightgray,      % Background color
    mathescape=true,         % Allow LaTeX math mode in code
    breaklines=true,        % Enable line breaking
    breakanywhere=true,     % Break lines anywhere
    samepage=false          % Explicitly allow page breaks
}
%%%%%%%%%%%%%%%%%%%%%%%%%%%%%%%%%  %%%%%%%%%%%%%%%%%%%%%%%%%%%%%%%%%
\usepackage[font=small,labelfont=bf]{caption}
\usepackage[
    backend=biber, 
    style=authoryear,
    hyperref=true
    ]{biblatex}
\usepackage{csquotes}
%\usepackage{biblatex}
\addbibresource{LaTex/chapters/references.bib}
%\usepackage[all]{hypcap} %Positioning of linked objects after jump
\setlength\parindent{0pt} %No intendation across whole doc